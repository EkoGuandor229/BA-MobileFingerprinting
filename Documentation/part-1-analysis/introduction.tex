\section{Einleitung}

\subsection{Problemstellung}
Die Erkennung und Verfolgung von Mobilgeräten anhand ausgesendeter Probe-Requests
ist in der Industrie eine bekannte Praxis.
Früher konnte ein Mobilgerät anhand der in Probe-Requests ausgesendeten MAC-Adresse
mit einem Fingerabdruck versehen werden und über grössere Distanzen und längere 
Zeit verfolgt werden. 
Seit einigen Betriebssystemversionen wird die MAC-Adresse aber in Probe-Requests 
zufallsgeneriert und kann nicht mehr für ein Fingerprinting verwendet werden.
Die Frage, welche in dieser Arbeit beantwortet werden soll, ist: 
Senden Mobilgeräte in Probe-Requests genügend Informationen aus, um damit 
einen Fingerabdruck zu generieren und lassen sich Mobilgeräte anhand dieser 
Informationen unterscheiden und verfolgen.
Spezifisch soll ein Verfahren entwickelt werden, welches erlaubt, Mobilgeräte 
im Empfangsbereich eines WLAN-Access-Point zu zählen.
Um die Frage zu beantworten, sollen Messungen mit Mobilgeräten durchgeführt werden,
um Gerätespezifisches Verhalten zu erkennen, welches für ein Fingerprinting 
verwendet werden kann.

\subsection{Herausforderungen}
Die grösste Herausforderung ist die Beschaffung der Mobilgeräte, die für die 
Messungen benötigt werden.
Um statistisch relevante Ergebnisse erzeugen zu können, sollte eine Vielzahl von 
Messungen mit unterschiedlichen Geräteherstellern, -typen und -Betriebssystemversionen
durchgeführt werden können.
Weiterhin sollten die Messungen in einer Umgebung durchgeführt werden, in der 
möglichst keine Störsignale aufgezeichnet werden. \\
Zusätzlich sollten die Messungen in einem möglichst realitätsnahen Umfeld durchgeführt
werden, was in einer Störungsfreien Umgebung nicht möglich ist.

Eine weitere Herausforderung ist das Zeitmanagement. 
Werden mehr Messungen durchgeführt, bleibt weniger Zeit, um an einem Prototyp zu 
arbeiten. 
Wenn ein Verfahren in der Analyse vielversprechend ist, in der Umsetzung aber 
nicht funktioniert kann dadurch zusätzlich Zeit verloren gehen.

\clearpage 

\subsection{Vorarbeit}
In einer Vorarbeit im Herbstsemester 2019/2020 wurde mit einem Machine-Learning-
Verfahren gearbeitet, welches mit Daten aus der Industrie trainiert wurde.
Insgesamt wurden 175 Millionen Probes in der Datensammlung analysiert und 
ein Prototyp entwickelt der gemäss der Dokumentation die Anzahl Passagiere 
in einem Bus mit $94\%$ Genauigkeit voraussagen kann. 

Die grössten Herausforderungen in der Vorarbeit bestanden darin, dass der Prototyp 
neben Mobilgeräten auch weitere Geräte im WLAN erkennt und diese nicht 
Filtern kann und dass Geräte vom selben Hersteller schwierig zu unterscheiden sind.

Vor allem das zweite Problem konnte in den Messungen dieser Arbeit \\ beobachtet und
bestätigt werden. 

Da in der Vorarbeit mit Datensätzen gearbeitet wurde, welche keine 
Angaben haben, welcher Probe-Request von welchem Gerät ausgesendet wurde, 
können diese Daten nicht wiederverwendet werden. 
\clearpage