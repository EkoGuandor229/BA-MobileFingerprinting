\cleardoublepage

\section{Android-Analyse
\label{section:androidanalysis}}
Android implementierte erstmals mit Android 6 (Marshmallow) Support für Developer
für die Verschleierung von MAC-Adressen in Probe-Requests.

In Versuchen im Paper "A Study of MAC Adress Randomization in Mobile Devices 
and when it Fails" von Jeremy Martin et. al. aus dem Jahr 2017 wird beschrieben, 
dass die meisten Geräte mit Android 6 die Randomisierung der MAC-Adresse auf Hardwareebene 
gar nicht unterstützen da die Chipsets nicht dazu in der Lage waren.

Erst ab Android 8 (Oreo) wurde die Randomisierung der MAC-Adressen bei Probe-Requests
von Android Geräten offiziell als Funktion implementiert (zuvor war die Option nur 
für Developer verfügbar) und standardmäßig verwendet.

Die Versionsgeschichte des Android Betriebssystem ist in der 
Tabelle~\ref{table:androidversionhistory} ersichtlich.

\subsection{Versionsgeschichte}
    \begin{table}[H]
        \begin{tabularx}{\linewidth}{XXXXX}
            \toprule 
            \textbf{Name} & \textbf{Nummer} & \textbf{Erscheinung} & \textbf{Supported} & \textbf{API Lvl} \\
            \midrule
            Honeycomb & 3.0 - 3.2.6 & 22.02.2011 & No & 11 - 13 \\
            Icecream Sandwich & 4.0 - 4.0.4 & 18.102011 & No & 14 - 15 \\
            Jelly Bean & 4.1 - 4.3.1 & 09.07.2012 & No & 16 - 18 \\
            KitKat & 4.4 - 4.4.4 & 31.10.2013 & No & 19 - 20 \\
            \rowcolor{lightgray}
            Lollipop & 5.0 - 5.1.1 & 12.11.2014 & No & 21 - 22 \\
            \rowcolor{lightgray}
            Marshmallow & 6.0 - 6.0.1 & 05.10.2015 & No & 23 \\
            \rowcolor{lightgray}
            Nougat & 7.0 - 7.1.2 & 22.08.2016 & No & 24 - 25 \\
            \rowcolor{lightgray}
            Oreo & 8.0 - 8.1 & 28.08.2017 & Yes & 26 - 27 \\
            \rowcolor{lightgray}
            Pie & 9 & 06.08.2018 & Yes & 28 \\
            \rowcolor{lightgray}
            Android 10 & 10 & 03.09.2019 & Yes & 29 \\
            \rowcolor{lightgray}
            Android 11 & 11 & 08.09.2020 & Yes & 30 \\
            \bottomrule 
        \end{tabularx}
        \caption{Versionsgeschichte des Android, 
        alle grau eingefärbten Zeilen benutzen MAC-Randomisierung.
        Supported Devices erhalten noch aktuelle Sicherheitsuptades
        \label{table:androidversionhistory}}
    \end{table}

    \clearpage

\subsection{Android 8 - Oreo}
Ab Android 8.0 verwenden Android-Geräte bei der Suche nach Wi-Fi Netzwerken
zufällige MAC-Adressen in Probe-Requests.
Für jeden Scan wird eine neue zufällige Adresse generiert und zusätzlich
wird die Sequenznummer zufällig generiert.

\subsection{Android 9 - Pie}
Mit Android 9.0 wurde die Developer-Option hinzugefügt,
für jede Verbindung mit einem Wi-Fi-Netzwerk eine zufällige MAC-Adresse zu verwenden.

\subsection{Android 10}
Die Verwendung von zufälligen MAC-Adressen für jedes Netzwerk wurde in Android 10 
standardmäßig aktiviert. 

\subsection{Android 11}
In Android 11 Beta 1 wurde die Option "Wi-Fi enhanced MAC Randomization" hinzugefügt.
Wenn der Access Point dies erlaubt, bzw. wenn auf dem Access Point die MAC-Randomisierung
aktiviert ist, dann wird jedes Mal, wenn sich das Gerät mit diesem Access Point verbindet,
eine neue MAC-Adresse generiert.

\clearpage