\section{Verwandte Arbeiten}
Es gibt duzende Papers zum Thema Fingerprinting auf Mobilgeräten.
Die relevantesten vier Arbeiten wurden in den folgenden Unterabschnitten
in der Reihenfolge ihrer Publikation zusammengefasst.

Es gilt zu erwähnen, dass in diesem Abschnitt Informationen aus den Papers
direkt übersetzt und übernommen wurde. 

\subsection{How Talkative is Your Mobile Device?}
\begin{itemize}
    \item Name: How Talkative is your Mobile Device? An Experimental Study of Wi-Fi 
    Probe-Requests
    \item Autoren: Julien Freudinger
    \item Publikationsjahr: 2015
\end{itemize}
Das Paper beschäftigt sich mit der Frage, welche Informationen von einem
Mobilgerät über Probe-Requests versendet werden.
Die wichtigsten Erkenntnisse sind eine Evaluation für den experimentellen
Versuchsaufbau, um Probe-Requests zu sniffen, die Messresultate aus den 
durchgeführten Messungen und eine Evaluation eines kommerziell genutzten 
MAC-Address-\\ Randomization-Algorithmus.

\subsubsection*{Experimenteller Versuchsaufbau und Resultate}
Die Experimente, welche im Paper beschrieben werden, 
beschreiben acht unterschiedliche Konfigurationen für die Messeinrichtung und
weitere acht für die Mobilgeräte und die aus den Messungen gewonnenen Resultate.
Für die Messungen wurden ein Samsung S3 mit Android 4.4.2, 
ein iPhone 6 mit IOS 8.1.3 und ein Nexus 5 mit Android L 5.0.1 verwendet.

\clearpage 

Die Konfigurationen für die Messeinrichtung sind nachfolgend aufgelistet.
Die Kanäle 1, 6 und 11 werden nachfolgend als reservierte Kanäle bezeichnet.
\begin{itemize}
    \item 1.dynamic: Eine Antenne, welche über die drei reservierten 2.4-GHz-Kanäle misst.
    \item 1.static: Drei Mess-Konfigurationen, jeweils statisch auf einem separaten reservierten 
    2.4-GHz-Kanal.
    \item 3.dynamic: Drei Antennen, die koordiniert über die reservierten 2.4-GHz-Kanäle 
    messen. (Die Antennen springen in regelmässigen Abständen auf den nächsten Kanal und 
    sind jeweils um einen Kanal verschoben)
    \item 3.dynamic.s: Drei Antennen, die über die reservierten 2.4-GHz-Kanäle messen.
    (Jede Antenne springt frei über die reservierten Kanäle.)
    \item 3.dynamic.sn: Drei Antennen die über die reservierten 2.4-GHz-Kanäle und die 
    in der Tabelle~\ref{table:higherfrequencyband} erwähnten 5-GHz-Kanäle messen.
    \item 3.static: Pro reserviertem Kanal wird jeweils eine Antenne statisch eingestellt.
\end{itemize}

Die Messungen wurden jeweils eine Stunde lang durchgeführt und fünfmal 
wiederholt.
Als Ergebnisse wurden die durchschnittlich aufgezeichnete Anzahl der Probe-
Requests und eine Schätzung der nicht aufgezeichneten Requests gespeichert.
Die nicht aufgezeichneten Probe-Requests können aus den Sequenznummern 
abgeschätzt werden. Werden in einer Messung beispielsweise drei Probe-Requests mit den
Sequenznummern 1, 3 und 5 aufgezeichnet, kann davon ausgegangen werden,
dass zwei Probes mit den Sequenznummern 2 und 4 nicht aufgezeichnet wurden.

Die Verwendung von mehreren Antennen ist ein Ansatz, welcher in anderen Arbeiten
z.T. nicht beachtet wurde und dazu führen kann, dass nur ca. $57$ Prozent der Probe
Requests tatsächlich gemessen und in den Ergebnissen berücksichtigt wurden.

\clearpage

Die Konfiguration der Mobilgeräte ist nachfolgend aufgelistet:
\begin{itemize}
    \item Default: Das Mobilgerät ist gesperrt, an ein Ladekabel angeschlossen,
    nicht mit einem WLAN verbunden und Bluetooth ist ausgeschaltet. Das Gerät
    kennt vier Netzwerke.
    \item Not Charging: Das Mobilgerät ist nicht an ein Ladekabel angeschlossen.
    \item Screen ON: Das Gerät wird alle fünf Minuten entsperrt und wieder gesperrt.
    \item Wi-Fi Connected: Das Gerät ist mit einem WLAN verbunden.
    \item Wi-Fi Settings ON: Das Mobilgerät befindet sich im Menu für die Auswahl 
    des WLANs, es wird aber sonst nicht mit dem Gerät interagiert.
    \item Bluetooth ON: Bluetooth ist während den Messungen eingeschaltet.
    \item Airplane ON: Der Flugmodus ist während den Messungen eingestellt und
    das WLAN auf dem Mobilgerät eingeschaltet.
    \item Known In Proximity: Ein dem Gerät bekanntes WLAN erscheint alle fünf
    Minuten für 10 Sekunden im Empfangsbereich des Mobilgeräts.
\end{itemize}
Die Resultate zeigen, dass wenn die WLAN-Einstellungen eingeschaltet werden 
oder wenn der Bildschirm aktiviert wird, die Anzahl ausgesendeter
Probe-Requests ansteigt. 
Eine Ausnahme ist hierbei das Nexus 5, welches keinen signifikanten Unterschied
bei den ausgesendeten Probes aufweist.

Weiterhin wurde gemessen, welche Auswirkung die Anzahl bekannter Netzwerke 
auf einem Mobilgerät auf die Anzahl ausgesendeter Probe-Requests hat.
Das iPhone sendet erst ab 20 bekannten SSIDs eine erhöhte Anzahl von Probes aus.
Auf dem Nexus werden unabhängig von den bekannten Netzwerken immer etwa gleich
viel Probe-Requests ausgesandt.
Lediglich das Samsung S3 versendet mit steigender Anzahl bekannter Netzwerke
eine erhöhte Anzahl Probes.

Im Paper wird jeweils davon ausgegangen, dass diese aufgezeichneten 
Geräte-verhalten vom Betriebssystem abhängen. 
Es wurden keine Messungen mit unterschiedlichen Geräten mit demselben
Betriebssystem vorgenommen, um den Einfluss von Hardware auf das Probing-
Verhalten zu untersuchen.

\clearpage

\subsubsection*{MAC-Address-Randomization-Algorithmus}
In den durchgeführten Messungen werden verschleierte MAC-Adressen des iPhones
untersucht und festgestellt, dass die Sequenznummern im iOS 8 nicht zufällig
initialisiert werden, sondern aufsteigend vorkommen.
Ausserdem wird in iOS 8 die MAC-Adresse nur dann verschleiert, wenn das Gerät 
im Ruhemodus ist. 
Das bedeutet, dass iPhones mit iOS 8 relativ einfach identifiziert werden können,
wenn man die nicht randomisierte MAC-Adresse aufzeichnet und die 
Sequenznummern mit denjenigen Probe-Requests vergleicht, die eine verschleierte
MAC-Adresse haben. Hat man beispielsweise in Wireshark einen Probe mit der 
MAC `Apple\_12:84:05' und der Sequenznummer `1337' und einen Probe-Request 
mit der MAC `52:61:6E:64:6F:6D' und der Sequenznummer `1340', kann man davon 
ausgehen, dass die beiden Probe-Requests vom selben Gerät ausgesendet wurden.

Weiterhin senden Mobilgeräte mit iOS 8 oder Android 4 und 5 in den Information-
Element-Feldern herstellerspezifische Informationen mit, welche es zusätzlich
vereinfachen, Probe-Requests mit verschleierten MAC-Adressen zu einem spezifischen
Gerät zuzuordnen.

\clearpage

\subsection{Defeating MAC Address Randomization Through Timing Attacks}
\begin{itemize}
    \item Name: Defeating MAC Address Randomization Through Timing Attacks
    \item Autoren: Célestin Matte et al.
    \item Publikationsjahr: 2016
\end{itemize}
In diesem Paper wird ein Verfahren erarbeitet, mit dem die Verschleierung
von MAC-Adressen mittels eines Timing-Angriffs neutralisiert wird.
Es wird davon ausgegangen, dass die verschiedenen Betriebssysteme dahingehend
angepasst wurden, dass die Mobilgeräte keine identifizierenden Informationen 
mehr aussenden und nur das Timing der Probe-Requests für ein möchliches 
Fingerprinting verwendet werden kann.
Dabei wird vorausgesetzt, dass die Frames, die von Geräten ausgesendet
werden, einem Muster folgen, welches sich für eine Identifikation nutzen lässt.

MAC-Adressen werden in vielen Implementationen nach einigen Probe-Requests 
neu randomisiert. 
Eine Gruppe von Probe-Requests, die in einem Zeitfenster von ungefähr zehn
Millisekunden ausgesandt werden, wird auch Burst genannt.
Ausserdem ist in der Praxis die tatsächliche Anzahl Mobilgeräte im Empfangsbereich
eines WLAN-Access-Points nicht bekannt.
Die Schwierigkeit eines solchen Ansatzes besteht darin, mit einer sehr geringen Menge 
von Informationen ein zuverlässiges Fingerprinting durchführen zu können, 
ohne dass zuvor eine Datensammlung möglich gewesen ist. 
Die Autoren beschreiben als Lösung für diese Problematik die Verwendung 
inkrementeller Lernmethoden für die Clusterbildung unter der Verwendung
einer handelsüblicher Netzwerkkarte, die nur einen Kanal gleichzeitig \\
überwachen kann.

\subsubsection*{Zeitbasierte Signatur}
Probe-Requests lassen sich gruppieren, indem die Inter-Frame-Arrival-Time
(Zwischenankunftszeit, nachfolgend IFAT genannt) der einzelnen Frames mit 
der selben MAC-Adresse analysiert wird.
Der Algorithmus~\ref{algorithm:timeattack} beschreibt in Pseudocode, 
wie ein Identifikator zu einem Frame hinzugefügt werden kann.

\clearpage

\begin{algorithm}
    \KwIn{
        \\
        $G$: groups of burdst sets, grouped by MAC
        \\
        $t$: distance threshold 
        \\
        $d$: a distance function
    }
    \KwResult{$A$: dictionary of aliases}
    \BlankLine
    $A \longleftarrow \emptyset $\;
    $D \longleftarrow \emptyset $\tcp*[r]{Database of signatures}
    \ForEach{$B \in G$}{
        $S \longleftarrow $signature($B$)\;
        $d_{\text{min}} \longleftarrow min(d(S,S')\, \text{where}\, S' \in D)$\;
        \If{$d_{\text{min}} < t$}{
            $A[B.mac] \longleftarrow A[S'.mac]$\;
        }\Else{
            $A[B.mac] \longleftarrow B.mac$\;
        }
        $D \longleftarrow D \cup S$
    }
    \Return{$A$}
    \caption{Algorithmus für die identifikation von MAC-Frames
    \label{algorithm:timeattack}}
\end{algorithm}

Der Algorithmus hat als Input ein beliebiges Capture-File mit gemessenen
Probe-Requests und liefert als Output ein Mapping von Identifikatoren und
den Frames. 
Für alle Frames mit der selben MAC-Adresse wird die IFAT, 
die Auftretenswahrscheinlichkeit und der Durchschnitt der IFAT berechnet.
Anhand dieser Berechnungen wird basierend auf der eingegebenen Distanzgrenze
(distance threshold) entschieden, ob zwei Signaturen zum selben Gerät gehören.

Im Paper werden drei Distanzmass-Algorithmen genannt, 
auf die hier nicht weiter eingegangen wird. 
Bei Interesse des Lesers empfiehlt sich das Studium des Papers.
Weiterhin wird der Algorithmus an einem Datenset von 120'000 Proberequests
validiert und hat eine Erfolgsquote von ungefähr $77,2$ Prozent.

\clearpage 

\subsection{Why MAC Address Randomization is not Enough}
\begin{itemize}
    \item Name: Why MAC Address Randomization is not Enough: 
    An Analysis of Wi-Fi Network Discovery Mechanisms
    \item Autoren: Mathy Vanhoef et al.
    \item Publikationsjahr: 2016
\end{itemize}

In der Recherche zum Thema Mobile Fingerprinting taucht dieses Paper in 
sehr vielen anderen Quellen als Referenz auf. 
Darin wird eine fundierte Analyse diverser Methoden für die Deanonymisierung
von Mobilgeräten vorgenommen und erwiesen, dass die Verschleierung von 
MAC-Adressen allein keine Garantie für die Privatsphäre der Nutzer ist.

Der Hauptansatz besteht in der Analyse von Information-Element-Feldern
und den Sequenznummern, um Mobilgeräte voneinander zu unterscheiden.
Insgesamt werden zwölf IE-Felder genannt, die für ein Fingerprinting 
relevant sein können und nachfolgend aufgelistet sind.
\begin{itemize}
    \item HT capabilities info 
    \item Ordered list of tags numbers 
    \item Extended capabilities
    \item HT A-MPDU parameters 
    \item HT MCS set bitmask
    \item Supported rates 
    \item Interworking access net. type
    \item Extended supported rates 
    \item WPS UUID 
    \item HT extended capabilities
    \item HT TxBeam Forming Cap. 
    \item HT Antenna Selection Cap.
\end{itemize}
Weiterhin könnte die Reihenfolge der IE-Tags eine weitere Informationsquelle 
für ein mögliches Fingerprinting sein.

\clearpage

Auch in diesem Paper wird ein Algorithmus beschrieben, 
welcher ein Fingerprinting auf Mobilgeräte ermöglicht.

\begin{algorithm}
    \KwIn{
        \\
        $P$: List of captured probe requests
        \\
        $\Delta T$: maximum time between two probes 
        \\
        $\Delta S$: maximum sequence number distance 
    }
    \KwResult{Set of clusters corresponding to devices}
    \BlankLine
    $M \longleftarrow \emptyset$\tcp*[r]{$M$ maps fingerprints to clusters}
    \ForAll{$p \in P$}{
        $f \longleftarrow \text{fingerprint}(p) $\tcp*[r]{Calculate IE fingerprint}
        $M[f].\text{append}(p) $\tcp*[r]{Append probe to cluster}
    }
    $ D \longleftarrow []$\tcp*[r]{List of clusters representing devices}
    \ForAll{$C \in M$}{
        $S \longleftarrow [] $\tcp*[r]{Will contain subdivision of C}
        $m \longleftarrow \text{max}(p.seq\, \mathbf{for}\, p\, \mathbf{in}\, C) $\;
        \ForAll{$p \in C$}{
            $\text{Find}\, i\, \text{such that:} $\tcp*[f]{Fing matching cluster}\\
            $\qquad \text{d}(S[i].last.seq,\, p.seq,\, m) \leq \Delta S$\;
            $\qquad \text{and}\, p.time - S[i].last.time \leq \Delta T$\; 
            \If{$no\, i\, found$}{
            $i \longleftarrow |S|$\tcp*[r]{Create new subcluster}
        }
        $S[i].append(p)$\tcp*[r]{Add p to subcluster}
        }
        $D.extend(S)$\tcp*[r]{Extend list D with S}
    }
    \Return{$D$}
    \caption{Gruppierungsalgorithmus basierend auf IE-Feldern
    \label{algorithm:ieattack}}
\end{algorithm}
    
Der Algorithmus berechnet zuerst für alle Probe-Requests einen Fingerprint
anhand der IE-Felder und weist diesen einem Cluster zu. 
Die Cluster werden basierend auf den Sequenznummern in Gruppen unterteilt.

Da in modernen Betriebssystemen die Sequenznummer auch zufällig gesetzt wird,
kann das Clustering nicht mehr anhand der Sequenznummer durchgeführt werden.
In einem eigenen Prototypen müsste für die Clusterbildung eine eigene
Methodik entworfen werden.

\clearpage

\subsection{Noncooperative 802.11 MAC Layer Fingerprinting}
\begin{itemize}
    \item Name: Noncooperative 802.11 MAC Layer Fingerprinting and Tracking
    of Mobile Devices
    \item Autoren: Pieter Robyns et al.
    \item Publikationsjahr: 2017
\end{itemize}

In diesem Paper wird eine pro-Bit-Entropie Analyse von Probe-Request-Frames
durchgeführt und eine Technik vorgestellt, die mittels eines Fingerprints 
und zeitlichen Daten ein Gerät mit bis zu $80$ Prozent Genauigkeit erkennen und 
verfolgen kann, ohne dass dafür spezielle Hardware benötigt wird.

\subsubsection*{Identifikatoren und Fingerprinting}
Es wird zwischen zwei Typen von Identifikatoren unterschieden.
Explizite Identifikatoren, zu denen eine MAC-Adresse gehört, erlauben es,
ein Gerät eindeutig zu identifizieren.
Implizite Identifikatoren sind Informationen, die nicht absichtlich für 
die Identifikation gedacht sind, aber sich von Gerät zu Gerät genügend 
unterscheiden, um damit ein Fingerprinting durchzu-führen.
Unter der Annahme, dass die MAC-Adresse in Mobilgeräten anonymi-siert wird,
müssen für ein Fingerprinting die impliziten Identifikatoren verwendet werden.

\subsubsection*{Bitweise MAC-Header-Analyse}
In weiteren Arbeiten wird oftmals ein spezifisches Element des MAC-Headers
für ein Fingerprinting verwendet und das restliche Frame verworfen.
Dieses Paper schlägt vor, anhand der Entropie der einzelnen Header-Felder
automatisch zu evaluieren, welche Informationen für ein Fingerprinting 
verwendet werden sollen.
Dabei wird davon ausgegangen, dass das Mobilgerät jedes Frame einzeln
anonymisiert und nicht mit einem Access Point assoziiert ist. 
Das bedeutet, die vorgeschlagene Technik ist für die Anwendung auf 
ein einzelnes Frame ausgelegt.

Im Paper werden drei wichtige Metriken vorgestellt:
\begin{itemize}
    \item Bit-Variabilität: Einzelne Bits unterscheiden sich von Gerät zu
    Gerät dahingehend, dass sie sich für ein Fingerprinting eignen.
    \item Bit-Stabilität: Für ein Gerät bleiben die Bits, 
    die sich zu anderen Geräten unterscheiden, in nacheinander auftretenden
    Frames soweit stabil, dass sie weiterhin für einen Fingerprint 
    verwendet werden können.
    \item Bit-Verwendbarkeit: Die Verwendbarkeit ist eine Kombination der
    Variabilität und der Stabilität und sagt aus, ob das Bit für ein 
    Fingerprinting tatsächlich verwendet wird.
\end{itemize}

\clearpage

\subsubsection*{Berechnung der Variabilität}
Die Variabilität ist die Entropie eines Bits gemessen über mehrere Frames
welche jeweils von einem anderen Mobilgerät ausgesendet werden.
Dabei wird pro Gerät nur ein einzelnes Frame in Betracht gezogen, 
um keinen Bias in das System einzufügen.

Um die Entropie eines Bits an der Position $i$ in einem Frame zu berechnen,
wird zuerst die diskrete Wahrscheinlichkeitsdichte ($P(X_{i})$) für 
jeden Bitwert an der Position $i$ berechnet.
$X_{i}$ ist dabei die Zufallsvariable, welche einen Bitwert repräsentiert.
Somit ist $X_{i} \in \{0, 1, U\}$, wobei $U$ bedeutet, dass das Bit nicht 
vorhanden ist.
Die Wahrscheinlichkeitsdichte kann dann verwendet werden, um die 
Shannon-Entropie zu berechnen, wie in der 
Formel~\ref{equation:shannonentropy} dargestellt.

\begin{equation}
    \label{equation:shannonentropy}
    H(X_{i}) = - \sum_{x \in \{0, 1, U\}} P(X_{ix}) \log_{3} P(X_{ix})
\end{equation}

Da für die Berechnung der Entropie ein Bit mit drei möglichen Zuständen
(Wert 1 oder 0 und nicht vorhanden) beschrieben wird, wird der $\log_{3}$
anstatt dem üblichen $\log_{2}$ verwendet.
$H(X_{i})$ hat nun einen Wert zwischen $0$ und $1$, wobei $0$ bedeutet, 
dass keine Entropie vorhanden ist und $1$ bedeutet, dass eine maximale 
Entropie vorhanden ist.
Das Resultat kann als Vektor $\vec{v}$ repräsentiert werden:

\begin{equation}
    \vec{v} = [H_{1}\; H_{2}\; \cdots H_{n-1}\; H_{n}]
\end{equation}

Hierbei ist $n$ die Anzahl der Bits im Frame und $H_{n}$ die Entropie des 
jeweiligen Bits ist.

\subsubsection*{Berechnung der Stabilität}
Die Stabilität eines Bits wird als $1$ minus die Entropie dieses Bits, 
gemessen über mehrere Frames von einem Gerät, definiert.
Der Wert sagt aus, wie hoch die Wahrscheinlichkeit ist, dass das 
Bit über mehrere Übertragungen gleich bleibt.
Der Nutzen der Stabilität ist, dass diejenigen Bits, die sich pro Gerät
häufig verändern, im Fingerprinting nicht berücksichtigt werden.
Die Formel~\ref{equation:stability} zeigt den Resultierenden 
Stabilitätsvektor $\vec{s_{d}}$ pro Mobilgerät $d$:

\begin{equation}
    \label{equation:stability}
    \vec{s_{d}} = [1-H_{d1}\; 1-H_{d2}\; \cdots 1-H_{dn-1}\; 1-H_{dn}]
\end{equation}

\clearpage

Im Gegensatz zur Variabilität bedeutet ein hoher Wert bei der Entropie,
dass die Stabilität geringer wird. 
Darum wird die Stabilität mit $1-H(X_{i})$ berechnet.
Da die Stabilität pro Gerät berechnet wird, muss der Vektor $\vec{s}$
durch eine Mittelwertsberechnung der Werte des Vektors $\vec{s_{d}}$
berechnet werden, um diesen für die weiteren Operationen verwenden zu können.

\subsubsection*{Berechnung der Verwendbarkeit}
Im Idealfall sind Variabilität und Stabilität beide genügend hoch, 
um damit ein Fingerprinting durchführen zu können. 
Um die Verwendbarkeit - darge-stellt durch den Vektor $\vec{u}$ - zu berechnen,
werden im Paper zwei Ansätze vorgestellt, die beide in der Praxis anwendbar
sind. 

Der erste Ansatz ist ein statistisches Vorgehen, welches die Variabilität
und Stabilität als Wahrscheinlichkeit betrachtet, da beide im Wertebereich
$[0, 1]$ sind. Davon ausgehend, dass $\vec{v} \text{ und } \vec{s}$ 
voneinander unabhängige Variablen sind, kann die Verwendbarkeit mit der
Formel
\begin{equation}
    \vec{u} = \vec{v} \odot \vec{s}
\end{equation}
berechnet werden.

Der zweite Ansatz geht davon aus, dass die Stabilität zu bevorzugen ist
und deshalb in der Berechnung nur die Variabilität verwendet wird.
Diejenigen Bits, deren Stabilität unter einem gewissen Schwellwert $\lambda$ 
liegen, werden aus der Berechnung ausgeschlossen.

\begin{equation}
    \vec{u_{i}} = 
    \begin{cases}
        \vec{v_{i}},\quad \text{if}\; \vec{s_{i}} \geq \lambda, \\
        0,\quad \text{otherwise}
    \end{cases}
\end{equation}

Falls beispielsweise der Schwellwert $\lambda = 1$ gesetzt wird, 
wird dadurch sichergestellt, dass kombinierte Bits im Fingerprint über die 
Zeit immer stabil bleiben.

Da die IE-Felder nicht unbedingt in allen Probe-Requests in der gleichen 
Reihenfolge vorkommen, müssen die Variabilität und Stabilität für 
die IE-Felder unabhängig ihrer Reihenfolge berechnet werden.

\clearpage