\begin{abstract}
Mobilgeräte senden für die Suche nach WLAN-Netzwerken Probe- \\ Requests aus.
In diesen Probe-Requests sind zusätzliche Informationen, wie beispielsweise 
die unterstützten Datenraten oder bekannte Netzwerke, enthalten.
Seit Android und iOS 8 werden MAC-Adressen in Probe-Requests randomisiert.

Ziel der Arbeit ist, das Verhalten von verschiedenen Mobilgeräten mit 
modernen Betriebssystemversionen zu analysieren und auf Basis 
der Erkenntnisse ein Programm zu entwickeln, welches Mobilgeräte von-einander 
unterscheiden kann. Die Unterscheidung kann in Form eines Fingerprintings 
vorgenommen werden und allenfalls auch für eine Verfolgung von bekannten Geräten
genutzt werden.

In der Arbeit wurden drei iOS-Geräte und neun Android-Geräte in insgesamt 108 
Einzelmessungen untersucht. Mit den Ergebnissen wurde ein Prototyp entworfen,
welcher Messungen aufgrund der zusätzlichen Felder in Probe-Requests filtern 
und die Gesamtzahl der Mobilgeräte im Empfangsbereich auswerten kann.

Ein Verfahren, mit dem man Mobilgeräte langfristig mit einem Fingerabdruck 
versehen kann, ist anhand der in den Messungen gewonnenen Erkenntnisse nicht 
umsetzbar. Es hat sich gezeigt, dass sich in den neueren Betriebssystemversionen 
die Probe-Requests nicht mehr wesen-tlich voneinander unterscheiden.

In künftigen Verfahren für die Erkennung, Unterscheidung und Verfolgung von 
Mobilgeräten wird deshalb auf weitere Informationsquellen wie die 
Bluetooth-Schnittstelle zurückgegriffen werden müssen.


\end{abstract}