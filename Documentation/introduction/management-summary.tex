\cleardoublepage

\section*{Kurzübersicht}

\subsection*{Situation}
Mobilgeräte verwenden für die Suche nach WLAN-Netzwerken sogenannte Probe-Requests.
Diese Nachrichten werden an alle Geräte im Sendebereich des Mobilgeräts ausgesandt
und enthalten oftmals weitere Informationen, wie beispielsweise bekannte 
Netzwerke oder unterstützte Datenraten.
Bis vor einigen Jahren wurde in diesen Probe-Requests die einzigartige 
Geräteadresse mitgesendet. Nun wird diese Adresse jeweils zufallsgeneriert.

Ein Verfahren, welches erlaubt, Mobilgeräte durch diese Probe-Requests zu 
erkennen, voneinander zu unterscheiden und über längere Zeit zu 
verfolgen, kann in vielen Anwendungen Gebrauch finden. Ein Beispiel währe 
die automatische Fahrgastzählung im moblilen Verkehr.


\subsection*{Vorgehen}
In dieser Arbeit wurden Messungen durchgeführt, um das Verhalten von Mobilgeräten
zu untersuchen. Anhand der Messergebnisse wurde ein Prototyp in Pyhton entwickelt,
welcher in der Lage ist, Mobilgeräte voneinander zu unterscheiden und die 
Anzahl Mobilgeräte im Empfangsbereich der Messantenne zu ermitteln.

\subsection*{Ergebnisse}
Die Auswertung der Messergebnisse hat ergeben, dass iOS-Geräte im Verhalten
beinanhe identisch sind. Es ist schwierig bis unmöglich, diese Geräte 
anhand ausgesendeter Probe-Requests zu unterscheiden. 
Android-Geräte haben im Verhalten noch genügend Unterschiede, dass eine 
Unterteilung möglich ist. Eine längere Verfolgung lässt sich aber auch 
nicht umsetzen.  

\subsection*{Ausblick}
Gerätehersteller verbessern mit jeder neuen Betriebssystemversion das Verhalten 
im Bezug auf die Erkennbarkeit und Unterscheidbarkeit. Zum einen weil gewisse 
Verfahren aufgrund neuer Normen und Privatsphäregesetze vorausgesetzt werden,
aber auch weil die Hersteller die eigenen Technologien für die Lokalisierung 
ihrer Geräte dadurch besser verkaufen können.

Es ist möglich, dass mit zusätzlichen Messungen auf weiteren Mobilgeräten 
weitere Verhaltensunterschiede erkannt werden können, die für eine Ver-besserung
der Verfahren verwendet werden können. Ausserdem kann mit hinzuziehen weiterer
Signalquellen - beispielsweise Bluetooth - die Unterscheidung von Mobilgeräten 
weiter verfeinert werden. 


