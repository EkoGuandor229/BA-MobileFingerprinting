\section{Abschliessendes Urteil}
In dieser Arbeit wurden Messungen durchgeführt, um das Probing-Verhal-ten von 
modernen Mobilgeräten und aktuellen Betriebssystemen zu ermitteln.
Anhand der Erkenntnisse aus diesen Messungen wurde ein Prototyp entwickelt, 
welcher aufgrund von aufgezeichneten Probe-Requests die Anzahl Mobilgeräte 
im Empfangsbereich berechnet. Dazu wurden Filterregeln in einem hierarchischen 
Filtervorgang implementiert. Der Prototyp wurde an dafür aufgezeichneten Messungen
mit mehreren paralell laufenden Ge-räten validiert und konnte die Anzahl Geräte
zum Teil korrekt ausgeben. Eine Validierung mit Messdaten aus der Praxis konnte 
nicht durchgeführt werden, da kein Datenset zur Verfügung stand, welches mehrere
Mobilgeräte beinhaltet aber auch die Gesamtzahl von Mobilgeräten erfasst hat.

Die in den Messungen aufgezeichneten Probe-Requests beinhalten zu wenige 
voneinander unterscheidbare Parameter, dass damit Geräte mit einem Fingerabdruck
versehen werden können. 
Fingerprinting von Mobilgeräten anhand von Probe-Requests wird mit neu 
erscheinenden Betriebssystemversionen immer schwieriger.
Es ist schon nicht mehr möglich iPhones mit iOS-14 voneinander zu unterscheiden 
und neuere Android-Versionen wird die randomisierung von Probe-Requests eher 
verbessern, als sie zu verschlechtern. 

Somit sind Probe-Requests künftig nicht mehr für Fingerprinting nutzbar.
Um bestehende Verfahren weiterhin nutzen zu können, müssen diese weiterentwickelt 
oder durch andere Verfahren ergänzt werden.