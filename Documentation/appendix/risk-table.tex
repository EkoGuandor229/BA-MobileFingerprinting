\chapter{Anhang G: Risikotabelle}
In den nachfolgenden Unterabschnitten sind der Projektplan, jeweils zum Beginn 
der Bachelorarbeit und zum Ende der Bachelorarbeit abgelegt.
\begin{landscape}
\begin{table}[h!]
    \tiny  
	\centering
	\begin{tabularx}{\linewidth}{l l l l l X X X}
        \toprule 
        \multirow{2}{*}{\textbf{Nr}} & \multirow{2}{*}{\textbf{Titel}} & \textbf{Maximaler} & \textbf{Eintrittswahr-} & \textbf{Gewichteter}   & \multirow{2}{*}{\textbf{Vorbeugung}} & \textbf{Verhalten beim}  & \multirow{2}{*}{\textbf{Risikoabdeckung }}\\
        & & \textbf{Schaden [h]} & \textbf{scheinlichkeit} & \textbf{Schaden} & & \textbf{Eintreten}  & \\
        \midrule 
        R1 & Testgeräte & 10 & 10\% & 1 & Genaue Planung, welche Geräte benötigt werden und wie diese beschafft werden können & Anpassen der Gerätespezifikationen und beschaffung über weitere Quellen (Ausleihen von Instituten, Kollegen, Familie) & Genaue Spezifikation der benötigten Geräte und -betriebssysteme. Frühe beschaffung durch Projektteilnehmer und -Betreuer \\ 
        \midrule 
        R2 & Testvorbereitung & 20 & 25\% & 5 & Genaue Spezifizierung von Testparametern, bevor der Versuchsaufbau stattfindet & Testspezifikationen müssen überarbeitet werden. & Verifizieren, dass Versuchsaufbau durchführbar ist und zu den erwarteten Ergebnissen führt. Abklären des Aufbaus mit Experten. \\ 
        \midrule 
        R3 & Testdurchführung & 40 & 15\% & 6 & Testgeräte genau auf den Versuchsaufbau vorbereiten, genaue Testspezifikationen. Sicherstellen durch periodische Überprüfung, dass laufende Tests die Betriebsparameter erfüllen & Der gesamte Test oder Teile davon müssen erneut durchgeführt werden. & Versuchsaufbau unter kontrollierten Bedingungen. Periodisches Überprüfen nach jedem Versuchsschritt, dass keine Fehler vorgekommen sind \\ 
        \midrule 
        R4 & Testauswertung  & 42 & 10\% & 4,2 & Erwartete Resultate in der Versuchsplanung definieren und bei der Durchführung kontrollieren & Versuchsaufbau anpassen und Tests wiederholen & Risiko lässt sich mitigieren, indem die Gerätespezifikation im vornherein sehr genau studiert wird und die erwarteten Ergebnisse in der Versuchsplanung definiert werden \\ 
        \midrule 
        R5 & Format Versuchsdaten & 60 & 15\% & 9 & Datenformat und Erwartete Resultate in der Testplanung spezifizieren. Vorbereiten der Testdokumentation & Versuche müssen wiederholt werden & Recherche, welche Formate und Speichermöglichkeiten sich am besten für die Versuche eignen. \\ 
        \midrule 
        R6 & Gerätemerkmale & 100 & 20\% & 20 & Recherchen, um vorab zu wissen, wie sich die Mobilgeräte verhalten & Abklären mit Betreuer, ob die Aufgabenstellung der BA an die Erkenntnisse angepasst werden muss. & Höhere Anzahl Messungen und genau spezifizierte erwartete Messergebnisse \\ 
        \midrule 
        R7 & Geräteverhalten & 40 & 15\% & 6 & Mehrere Messungen mit gleichem Gerätetyp und OS-Version, um Abweichungen zu erkennen. & Mehr Geräte organisieren und weitere Versuche anstellen & Messungen mit unterschiedlichen Gerätetypen und verschiedenen Betriebssystemversionen \\ 
        \midrule 
        R8 & Hardwareverhalten & 42 & 10\% & 4.2 & Mehrere Messungen mit identischem OS auf unterschiedlicher HW durchführen & Mehr Geräte organisieren und weitere Versuche anstellen & Mehrere Geräte mit identischem OS verwenden \\
		\bottomrule 
	\end{tabularx} 
	\caption{Risikotabelle Projektbeginn
	\label{table:riskbeginn}} 
\end{table}

\clearpage 

\begin{table}[h!]
    \tiny  
	\centering
	\begin{tabularx}{\linewidth}{l l l l l X X X}
		\toprule 
        \multirow{2}{*}{\textbf{Nr}} & \multirow{2}{*}{\textbf{Titel}} & \textbf{Maximaler} & \textbf{Eintrittswahr-} & \textbf{Gewichteter}   & \multirow{2}{*}{\textbf{Vorbeugung}} & \textbf{Verhalten beim}  & \multirow{2}{*}{\textbf{Risikoabdeckung }}\\
        & & \textbf{Schaden [h]} & \textbf{scheinlichkeit} & \textbf{Schaden} & & \textbf{Eintreten}  & \\
        \midrule 
        R1 & Testgeräte & 5 & 5\% & 0,25 & Genaue Planung, welche Geräte benötigt werden und wie diese beschafft werden können & Anpassen der Gerätespezifikationen und beschaffung über weitere Quellen (Ausleihen von Instituten, Kollegen, Familie) & Genaue Spezifikation der benötigten Geräte und -betriebssysteme. Frühe beschaffung durch Projektteilnehmer und -Betreuer \\ 
        \midrule 
        R2 & Testvorbereitung & 10 & 10\% & 1 & Genaue Spezifizierung von Testparametern, bevor der Versuchsaufbau stattfindet & Testspezifikationen müssen überarbeitet werden. & Verifizieren, dass Versuchsaufbau durchführbar ist und zu den erwarteten Ergebnissen führt. Abklären des Aufbaus mit Experten. \\ 
        \midrule 
        R3 & Testdurchführung & 30 & 10\% & 3 & Testgeräte genau auf den Versuchsaufbau vorbereiten, genaue Testspezifikationen. Sicherstellen durch periodische Überprüfung, dass laufende Tests die Betriebsparameter erfüllen & Der gesamte Test oder Teile davon müssen erneut durchgeführt werden. & Versuchsaufbau unter kontrollierten Bedingungen. Periodisches Überprüfen nach jedem Versuchsschritt, dass keine Fehler vorgekommen sind \\ 
        \midrule 
        R4 & Testauswertung  & 42 & 5\% & 2,1 & Erwartete Resultate in der Versuchsplanung definieren und bei der Durchführung kontrollieren & Versuchsaufbau anpassen und Tests wiederholen & Risiko lässt sich mitigieren, indem die Gerätespezifikation im vornherein sehr genau studiert wird und die erwarteten Ergebnisse in der Versuchsplanung definiert werden \\ 
        \midrule 
        R5 & Format Versuchsdaten & 60 & 5\% & 3 & Datenformat und Erwartete Resultate in der Testplanung spezifizieren. Vorbereiten der Testdokumentation & Versuche müssen wiederholt werden & Recherche, welche Formate und Speichermöglichkeiten sich am besten für die Versuche eignen. \\ 
        \midrule 
        R6 & Gerätemerkmale & 80 & 20\% & 16 & Recherchen, um vorab zu wissen, wie sich die Mobilgeräte verhalten & Abklären mit Betreuer, ob die Aufgabenstellung der BA an die Erkenntnisse angepasst werden muss. & Höhere Anzahl Messungen und genau spezifizierte erwartete Messergebnisse \\ 
        \midrule 
        R7 & Geräteverhalten & 30 & 15\% & 4,5 & Mehrere Messungen mit gleichem Gerätetyp und OS-Version, um Abweichungen zu erkennen. & Mehr Geräte organisieren und weitere Versuche anstellen & Messungen mit unterschiedlichen Gerätetypen und verschiedenen Betriebssystemversionen \\ 
        \midrule 
        R8 & Hardwareverhalten & 42 & 5\% & 2,1 & Mehrere Messungen mit identischem OS auf unterschiedlicher HW durchführen & Mehr Geräte organisieren und weitere Versuche anstellen & Mehrere Geräte mit identischem OS verwenden \\
		\bottomrule
	\end{tabularx} 
	\caption{Risikotabelle Projektabschluss
	\label{table:riskend}} 
\end{table}  

\clearpage 

\begin{table}[h!]
    \tiny  
	\centering
	\begin{tabularx}{\linewidth}{l X}
        \toprule 
        \textbf{Titel} & \textbf{Beschreibung}  \\
        \midrule 
        Testgeräte & Die erforderlichen Testgeräte (Smartphones/Access-Points) können nicht organisiert werden oder sind im Rahmen der Experimente nicht brauchbar. \\ 
        \midrule 
        Testvorbereitung & Die Testspezifikation ist fehlerhaft/mangelhaft und Experimente können nicht durchgeführt werden.  \\ 
        \midrule 
        Testdurchführung & Fehler, welche bei der Testdurchführung auftreten.  \\ 
        \midrule 
        Testauswertung & Der durchgeführte Test liefert keine signifikanten Resultate und es können keine Schlüsse aus dem Resultat gezogen werden.  \\ 
        \midrule 
        Format Versuchsdaten & Daten, die in den Versuchen gewonnen werden, lassen sich nicht weiter verwenden (falsches Format, ungenügende Resultatmenge)  \\ 
        \midrule 
        Gerätemerkmale & Mobilgeräte lassen sich nicht oder ungenügend anhand der ausgesendeten Probe Requests unterscheiden \\ 
        \midrule 
        Geräteverhalten & Geräte des selben Typs/OS verhalten sich nicht immer gleich. Schwierig/Unmöglich, herauszufinden, welche Faktoren ein unterschiedliches Verhalten begünstigen  \\ 
        \midrule 
        Hardwareverhalten & Hohe Komplexität, da sich OS je nach unterliegender Hardware verschieden verhalten können  \\
		\bottomrule 
	\end{tabularx} 
	\caption{Beschreibung zu Risikotabellen~\ref{table:riskbeginn} und~\ref{table:riskend}
	\label{table:riskdescription}} 
\end{table}
\end{landscape}



