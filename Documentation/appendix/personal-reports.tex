\chapter{Persönliche Berichte}
\cleardoublepage

\section*{Janik Schlatter}
Ich war vorerst nicht komplett von dem Thema überzeugt, da ich noch nie 
eine Forschungsarbeit gemacht habe und eher Praktisch veranlagt bin. 
Jedoch ist das Thema inhaltlich spannend und es hat mich thematisch interessiert.

Zu Beginn hatten wir einige Probleme mit dem Projektmanagement, da wir keine 
Ahnung hatten wie ein Projektplan für eine Forschungsarbeit aussehen muss. 
Danach lief es jedoch reibungslos und wir konnten bald mit dem Testen der 
Mobilgeräten beginnen.

Das Testen der Mobilgeräte in der Messkammer war sehr interessant, 
jedoch wurden wir durch Corona in der Benutzung der Kammer ein wenig 
eingeschränkt. Nach der ersten Testwoche waren einige unserer Hoffnungen 
erloschen, da das neue iOS Betriebssystem ihre Privacy im Griff hat. 
Nach den Android Messung wurden dann zum Glück doch noch ein Paar 
Ansätze ersichtlich wie wir die Sache angehen könnten.

Die Entwicklung des Prototyps war spannend, wir mussten bereits zu Beginn 
einige Entscheidungen treffen, beispielsweise wie wir die Daten aus dem 
Wireshark in das Python Programm bringen. Als wir dann zum Proof-of-Concept 
kamen mussten wir einige Filter-Methoden jedoch verwerfen, da sie keine 
anständigen Resultate lieferten. 

Die Zusammenarbeit mit Mike Schmid verlief reibungslos, 
da wir bereits an einigen Projekten an der HSR zusammengearbeitet
 haben und wir uns in unseren Stärken sehr gut ergänzen. Zusätzlich 
 stand uns Beat Stettler stehts mit Hardware und professioneller Beratung 
 zur Verfügung.

Abschliessend sind wir leider nicht zu einem Fingerprinting gekommen 
und ich denke nicht, dass es realistisch ist in Zukunft nur durch Probe-Requests 
ein Fingerprinting zu erreichen, da die neuen iOS und Android Versionen die 
Sicherheit generell immer verstärken. Trotzdem bin ich mit unserer Arbeit 
sehr zufrieden und wir konnten dennoch einiges erreichen und lernen.


\clearpage

\section*{Mike Schmid}
Da ich mich für Privacy- und Security-Themen interessiere, konnten wir mit der 
Zuteilung der Bachelorarbeit eine Arbeit durchführen, 
die in meinem Interessengebiet liegt.

Zum Thema "Mobile Fingerprinting" finden sich viele Paper und wir konnten uns 
in der Recherchephase einen guten Überblick verschaffen, welche Ansätze 
zielführend sind, und welche im Anbetracht der neuen 
Android/iOS-Betriebssystemversionen wahrscheinlich nicht angewandt werden können.
Die breite Unterstützung durch den Betreuer, das INS, das IFS und das ICOM ist mir
sehr positiv Aufgefallen und meiner Meinung nach wäre eine Durchführung der 
Messungen ohne diese Unterstützung nicht so angenehm verlaufen.

Die grössten Probleme traten mit der Projektplanung auf. 
Da weder Janik Schlatter noch ich jemals eine Forschungsarbeit betrieben haben,
konnten wir uns auf kenerlei bestehende Erfahrungen stützen. 
Dadurch haben wir uns vor allem im Aufwand für die Arbeitspackete oft verschätzt
und mussten im Verlauf der Arbeit den Meilenstein nach den Android-Messungen 
verschieben, da der zeitliche Aufwand für die Messdurchführung unsere schätzungen
deutlich übertroffen hat.

Die Zusammenarbeit mit Janik Schlatter war sehr angenehm. 
Wir haben bereits gemeinsam einige Projekte an der Hochschule durchgeführt und 
zu Beginn der Bachelorarbeit waren wir bereits so gut aufeinander Abgestummen,
dass während der Arbeit keine Probleme aufgetreten sind.
Weiterhin haben wir uns sehr gut in unseren Stärken/Schwächen und den 
jeweiligen Interessen für die einzelnen Tätigkeien ergänzt. 
Janik, eher der praktisch veranlagte Mensch, konnte sich um die Durchführung der 
Messungen und Umsetzung des Prototypen kümmern, während ich die Organisation 
und den theoretischen input in der Umsetzung liefern konnte.
Auch in den gemeinsamen Programmiersessionen konnten wir effizient die anstehenden
Probleme lösen und die in den Messungen gelernten Fakten praktisch umsetzen.

Mit den Ergebnissen bin ich weitestgehend zufrieden. Gerne hätte ich ein 
Fingerprinting und Tracking der Mobilgeräte praktisch umsetzen können, 
aber dadurch, dass  die verschleierung der MAC-Adressen in den neueren 
Betriebssystemversionen eine Unterscheidung von Geräten fast unmöglich macht,
konnten wir nach bestem Wissen keinen funktionierenden Ansatz umsetzen.
Allerdings bin ich aus Privacy-Sicht auch froh, 
dass Hersteller die Technologien weiterentwickeln und dadurch die 
Privatsphäre der Benutzer besser schützen können.  



\clearpage

