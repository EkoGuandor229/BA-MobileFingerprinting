\chapter{Anhang A: Abkürzungsverzeichnis}
In der Tabelle~\ref{table:abreviationsandglossary} sind die in der Arbeit 
verwendeten Abkürzungen und deren Beschreibung dokumentiert.

\clearpage

\begin{longtable}{|l|l|}
    % Header on first page
    \hline
    \multicolumn{1}{|c|}{\textbf{Abkürzung}} & \multicolumn{1}{c|}{\textbf{Bezeichnung}} \\
    \hline
    \endfirsthead
    % Header on consecutive pages
    \hline 
    \multicolumn{2}{|c|}{{Fortführung der Tabelle}} \\
    \hline
    \multicolumn{1}{|c|}{\textbf{Abkürzung}} & \multicolumn{1}{c|}{\textbf{Bezeichnung}} \\
    \hline 
    \endhead
    % Footer to notify about continuation
    \hline 
    \multicolumn{2}{|c|}{{Tabelle wird auf der nächsten Seite fortgeführt}} \\
    \hline
    \endfoot
    \endlastfoot
    avg. & Average. Durchschnitt (-wert) \\
    \hline 
    AVT & Arbeitsverwaltungstool der OST-Schweizer Fachhochschule. \\
    \hline
    BSSID \& SSID & (Basic) Service Set ID. Für die Zuordnung von Geräten zu \\
    & einem Netzwerk. \\
    \hline
    CSV & Comma Separated Values. Datenformat mit kommaseparierten Einträgen \\
    \hline
    DS Parameter & Direct Sequence Parameter. Beschreibt vom Netzwerk \\
    & verwendeten WLAN-Kanal. \\
    \hline
    ECTS & European Credit Transfer and Accumulation System.\\
    & Standard für die Vergabe von akademischen Credits. \\
    \hline
    GHz \& MHz & Giga- \& Megahertz. Einheit für Frequenz. \\ 
    & Ein GHz ist 1000 MHz und 1 MHz ist 1000 Hz. \\
    \hline
    HT & Higher Throughput. 802.11n Erweiterung für höhere \\ 
    & Datendurchsätze. \\
    \hline
    A-MPDU & Aggregate MPDU. Verfahren im 802.11n Standard für die \\
    & Aggregation von MPDUs für erhöhten Datendurchsatz. \\
    \hline
    ICOM & Institut für Kommunikationssysteme der OST-Schweizer \\
    & Fachhochschule.\\
    \hline
    IE-Felder & Information-Element-Felder. Zusätzliche Geräteparameter,\\
    & die in Probe-Requests mitgesendet werden. \\
    \hline
    IEEE & Institute of Electrical and Electrionics Engineers. \\
    & Berufsverband von Elektroingenieuren welcher verschiedene \\
    & Standards herausgebracht hat. \\
    \hline
    IBAT & Inter-Burst-Arrival-Time. Zwischenankunftszeit von Bursts. \\
    \hline
    IFAT & Inter-Frame-Arrival-Time. Zwischenankunftszeit von Frames. \\
    \hline
    IFS & Institut für Software der OST-Schweizer Fachhochschule. \\
    \hline
    INS & Institute for Networked Solutions der OST-Schweizer \\
    & Fachhochschule. \\
    \hline
    iOS & Apple Betriebssystem. \\
    \hline
    IP & Internet-Protocol. Netzwerkprotokoll und Grundlage des \\
    & Internets. \\
    \hline
    JSON & Javascript Object Notation. Datenformat für den \\ 
    & Datenaustausch zwischen Anwendungen. \\
    % Page Break 
    MAC & Media Access Control. Adresse für den Medienzugriff \\
    & in der Netzwerktechnik. \\
    \hline
    MPDU & MAC Protocol Data Unit. Nachricht, die zwischen \\
    & MAC-Entitäten in Form von Frames ausgetauscht wird. \\
    \hline
    NIC & Network Interface Controller. Netzwerkkontroller- \\
    & spezifische Kennung für die eindeutige Identifikation \\ 
    & von Netzwerkkarten. \\
    \hline
    Nof & Number of. Anzahl von Entitäten. \\
    \hline
    OSI & Openn Systems Interconnection. Konzeptionelles Modell \\
    & für die standardisierung eine Computer Systems. \\
    \hline
    OUI & Organisational Unique Identifier. Herstellerspezifische \\
    & Kennung in der MAC-Adresse. \\
    \hline
    PNO \& ePNO & (Enhanced) Preferred Network Offload. Apple \\
    & Spezifikation für die Erkennung bekannter Netzwerke. \\
    \hline
    TCP & Transmission Control Protocol. Netzwerkprotokoll für die \\
    & Kommunikation über das Internet. \\
    \hline
    U/L Bit & Universal/Local Bit. Synonyme Bezeichnung für das \\
    & lokale Bit. \\
    \hline
    WEP & Wired Equivalent Privacy. Wi-Fi Standard für die \\
    & Netzwerksicherheit. \\
    \hline
    Wi-Fi & Wireless Fidelity. Protokoll für kabellose Netzwerke. \\
    & Oft als Synonym für WLAN verwendet. \\
    \hline
    WLAN & Wireless Local Area Network. Bezeichnung für ein \\ 
    & Kabelloses Computernetzwerk. \\
    \hline
    WPS & Wireless Protected Setup. Netzwerkstandard für schnelle \\
    & Verbindungen mit WLAN. \\
    \hline
    UUID & Universally Unique Identifier. 128-Bit ID für die eindeutige \\
    & Identifikation von Computersytemen oder -anwendungen. \\
    \hline
    \caption{Abkürzungen}  \label{table:abreviationsandglossary} \\
\end{longtable}

\clearpage 